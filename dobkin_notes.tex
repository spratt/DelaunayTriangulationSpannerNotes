%%%%%%%%%%%%%%%%%%%%%%%%%%%%%%%%%%%%%%%%%%%%%%%%%%%%%%%%%%%%%%%%%%%%%%
% LatexTemplate
% February 10, 2013
% By Simon Pratt (mostly)
\documentclass{tufte-handout}
\usepackage{natbib}

%%%%%%%%%%%%%%%%%%%%%%%%%%%%%%%%%%%%%%%%%%%%%%%%%%%%%%%%%%%%%%%%%%%%%%
% Configuration
\fancyhead[L]{Dobkin Notes}
\fancyhead[R]{Simon Pratt}
\bibliographystyle{plain}

\title{Dobkin Notes}
\author{Simon Pratt}

%%%%%%%%%%%%%%%%%%%%%%%%%%%%%%%%%%%%%%%%%%%%%%%%%%%%%%%%%%%%%%%%%%%%%%
% Document
\begin{document}

\maketitle

\vspace{5mm}

% Content starts here

The Delaunay triangulation of a set of points in the plane is a
spanner with spanning ratio $c \le ((1 + \sqrt{5})/2)\pi \approx
5.08$.  This was proven in the paper ``Delaunay Graphs Are Almost as
Good as Complete Graphs'' by Dobkin, Friedman, and Supowit
\cite{Dobkin:1990}.

\section{Introduction}

Let $S$ be a set of points in the plane and $DT(S)$ be the edges of
the Delaunay triangulation of $S$.  Let the path along the Delaunay
edges be a \emph{Delaunay path}.

\section{One-Sided Path: The Easy Case}

If all edges along the Delaunay path are either all above or all below
the line connecting points $a,b \in S$, we say that this is a
one-sided path.

\section{The Harder Case}

Blah blah blah, blah blah blah blah.  Blah blah blah, blah blah blah
blah.  Blah blah blah, blah blah blah blah.

% Content ends here

\bibliography{references} % uncomment line to add references

\end{document}

%%%%%%%%%%%%%%%%%%%%%%%%%%%%%%%%%%%%%%%%%%%%%%%%%%%%%%%%%%%%%%%%%%%%%%
% LatexTemplate
% February 10, 2013
% By Simon Pratt (mostly)
\documentclass{tufte-handout}

\usepackage{CGAlgorithms}
\usepackage{QuestionAnswer}
\usepackage{TheoremStuff}
\usepackage{HeaderStuff}
\usepackage{natbib}

%%%%%%%%%%%%%%%%%%%%%%%%%%%%%%%%%%%%%%%%%%%%%%%%%%%%%%%%%%%%%%%%%%%%%%
% Configuration
\newcommand{\DocTitle}{Delaunay Triangulation Spanner Notes}
\newcommand{\DocAuthor}{Simon Pratt}

\title{\DocTitle}
\author{\DocAuthor}
\fancyhead[L]{\DocTitle}
\fancyhead[R]{\DocAuthor}
\bibliographystyle{plain}

%%%%%%%%%%%%%%%%%%%%%%%%%%%%%%%%%%%%%%%%%%%%%%%%%%%%%%%%%%%%%%%%%%%%%%
% Document
\begin{document}

\maketitle

% Content starts here

\part{Dobkin's Results}

The Delaunay triangulation of a set of points in the plane is a
spanner with spanning ratio $c \le ((1 + \sqrt{5})/2)\pi \approx
5.08$.  This was proven in the paper ``Delaunay Graphs Are Almost as
Good as Complete Graphs'' by Dobkin, Friedman, and Supowit
\cite{Dobkin:1987} \cite{Dobkin:1990}.

\section{Introduction}

Let $S$ be a set of points in the plane and $DT(S)$ be the edges of
the Delaunay triangulation of $S$.  Let a path along the Delaunay
edges be a \emph{direct DT path}.

\section{One-Sided Path: The Easy Case}

If all edges along the direct DT path between points $a,b \in S$ are
either all above or all below the line connecting $a,b$, we say that
this is a one-sided path.



\section{The Harder Case}

Blah blah blah, blah blah blah blah.  Blah blah blah, blah blah blah
blah.  Blah blah blah, blah blah blah blah.

\part{Keil's Results}

Blah blah blah, blah blah blah blah.  Blah blah blah, blah blah blah
blah.  Blah blah blah, blah blah blah blah.

% Content ends here

\newpage

\bibliography{references} % uncomment line to add references

\end{document}

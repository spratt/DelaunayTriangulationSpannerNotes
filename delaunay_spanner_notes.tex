%%%%%%%%%%%%%%%%%%%%%%%%%%%%%%%%%%%%%%%%%%%%%%%%%%%%%%%%%%%%%%%%%%%%%%
% DelaunayTriangulationSpannerNotes
% March 20th, 2013
% By Simon Pratt (mostly)
\documentclass{tufte-handout}

\usepackage{geometry}
\geometry{verbose,tmargin=1in,bmargin=0.5in,lmargin=.5in}
\usepackage{includes/CGAlgorithms}
\usepackage{includes/TheoremStuff}
\usepackage{includes/HeaderStuff}
\usepackage{natbib}
\usepackage[pdftex]{graphicx}

%%%%%%%%%%%%%%%%%%%%%%%%%%%%%%%%%%%%%%%%%%%%%%%%%%%%%%%%%%%%%%%%%%%%%%
% Configuration
\newcommand{\DocTitle}{Delaunay Graph Spanner Notes}
\newcommand{\DocAuthor}{Simon Pratt}

\title{\DocTitle}
\author{\DocAuthor}
\fancyhead[L]{\DocTitle}
\fancyhead[R]{\DocAuthor}
\bibliographystyle{plain}

%%%%%%%%%%%%%%%%%%%%%%%%%%%%%%%%%%%%%%%%%%%%%%%%%%%%%%%%%%%%%%%%%%%%%%
% Document
\begin{document}

\maketitle
\vspace{1cm}
% Content starts here

In these notes, we discuss the major results with respect to the
Delaunay Graph as a spanner.

\part{Delaunay Graph}

$P$ is a set of points in the plane, $DG(P)$ is a graph whose vertex
set is $P$ where $u$ and $v$ are connected by an edge only if the
voronoi regions for $u$ and $v$ share an edge.

\begin{figure}
  % http://groups.csail.mit.edu/graphics/classes/6.838/F01/lectures/Delaunay/Delaunay2D.ppt
  \includegraphics[scale=1.6]{figures/delaunay_graph.png}
  \caption{The Delaunay graph on $P$, including the boundaries of the
    Voronoi regions.}
\end{figure}

\newpage

\part{Dobkin's Results}

The Delaunay triangulation of a set of points in the plane is a
spanner with spanning ratio $c \le ((1 + \sqrt{5})/2)\pi \approx
5.08$.  This was proven in the paper ``Delaunay Graphs Are Almost as
Good as Complete Graphs'' by Dobkin, Friedman, and Supowit.  
\cite{Dobkin:1987} \cite{Dobkin:1990}

\section{Introduction}

We consider the path between two arbitray points $a,b \in P$.  Let the
line segment between $a$ and $b$ be the \emph{direct line}.  We
construct \emph{the direct DT path} by walking along the direct line,
each time a new face of the Voronoi diagram is reached we add the
corresponding edge in the Delaunay Graph.

\section{One-Sided Path: The Easy Case}

If all edges along the direct DT path between points $a,b \in P$ are
either all above or all below the direct line, we say that this is a
one-sided path.

\begin{figure*}
  \includegraphics[scale=0.8]{figures/one-sided_path.pdf}
  \caption{The cyan line shows the direct path, the green line shows
    the direct DT path, the dashed red lines show the boundaries of
    the Voronoi regions, and the circumcircles (also dashed) are
    blue.}
\end{figure*}

Without loss of generality, we can say that the line segment between
points $a$ and $b$ lies on the x-axis.

\begin{Lemma}

  Points along a direct DT path are monotonic in $x$.

\end{Lemma}

\begin{Lemma}

  All points along the direct DT path from $a$ to $b$ are contained
  within or on the boundary of the circle with $a$ and $b$
  diametrically opposed.
  
\end{Lemma}

\begin{Lemma}

  The boundary of a connected union of $n$ circles has length at most
  $\pi \cdot ( x_r - x_l )$ where $x_r$ and $x_l$ are the extreme x
  coordinates of any of the circles.
  
\end{Lemma}

\begin{proof}

  We prove by induction that the upper boundary of the circles has
  length at most $\frac{\pi}{2} \cdot ( x_r - x_l )$, from which the
  lemma follows by symmetry.
  
  In the case of a single circle, the upper boundary is half of the
  circumference of the circle:, $\frac{\pi}{2} \cdot ( x_r - x_l )$.
  The lemma holds for $k \ge 1$ circles, we now show that it holds for
  $k+1$ circles.

  Without loss of generality, we say that the $k+1$th circle is added
  at the right-most extremity of the $k$ circles.  

\begin{figure*}
  \includegraphics[scale=2.6]{figures/circle_unions.pdf}
  \caption{Let $B$ be the upper boundary of the $k$ circles.  Let $b'$
    be the length of $B$ contained within the $k+1$th circle.  Let $x$
    be the upper boundary of the new circle not contained within $B$,
    and $x'$ be the rest.  Let $y$ be the circle whose left-most point
    is the left-most point of the $k+1$th circle, and whose right-most
    point is the right-most point of $B$.  Let $z$ be the circle whose
    left-most point is the right-most point of $B$ and whose
    right-most point is the right-most point of the $k+1$th circle.}
\end{figure*}

From the inductive hypothesis, we know

\begin{itemize}

\item $B \le \frac{\pi}{2} \cdot (d + d')$

\item $z \le \frac{\pi}{2} \cdot d''$

\end{itemize}

Therefore $B - b' + x \le \frac{\pi}{2} \cdot (d + d' + d'') = B
+ z$.

So we wish to show: $x - b' \le z$.

We know:

\begin{itemize}

\item $y \le x' + b' \leftrightarrow y - x' \le b'$

\item $x' + x = y + z$

\end{itemize}

So

\begin{align*}
  %
  x + x' &= y + z \\
  x &= y + z - x' \\
  x &\le z + b' \\
  x - b' &\le z \\
  %
\end{align*}
  
\end{proof}

From lemmas 1 and 3, it follows that the one-sided path is at most
$\pi/2$ times as long as the euclidean distance between the endpoints.

The properties of the delaunay triangulation don't seem to allow for
any kind of zig-zag one-sided path.

\section{The Harder Case}

The direct DT path may cross the x-axis $\BigOmega{n}$ times,
which can yield a much longer path.

\begin{figure}
  \includegraphics[scale=1.0]{figures/two_sided_path_center_circles.pdf}
  \caption{A two-sided direct DT path showing the circles whose union
    forms $C$ and the dotted circle with $a,b$ diametrically
    opposed.}
\end{figure}

The general idea is that we stick to the region above the x-axis as
much as possible, and follow the path below the x-axis if the distance
travelled across the axis is small compared to the distance travelled
along the path before the next time the path crosses the x-axis.

\newpage

\begin{figure}
  \includegraphics[scale=2.9]{figures/height_width.pdf}
  \caption{Let $a = p_0, \ldots, p_i, \ldots, p_n = b$ be the direct
    DT path from $a$ to $b$.  For each pair $p_i, p_{i+1}$ create the
    circle on whose boundary these points lie, and whose centre is on
    the line segment between $a$ and $b$.  Let the union of these
    circles be $C$.

    Let $b_i$ be the last point before the direct DT
    path dips below the x-axis, let $b_j$ be the next point after
    $b_i$ on or above the x-axis.  Let $T$ be the section of $C$
    between $b_i$ and $b_j$.  Let $h = min \{ y(q): q \text{ lies on }
    T \}$, and $w = x(b_j) - x(b_i)$.

    Let $R_1$ be the region between $b_i$ and $b_j$ above the line
    $y=-h$.  Let $R_2$ be the region below the line $y=-h$.  Let $R_3$
    be the region above the line $y=-h$ and to the left of $b_i$.  Let
    $R_4$ be the region above the line $y=-h$ and to the right of
    $b_j$. }
\end{figure}

To be specific, we take the direct DT path only if $h \le w/4$.

Otherwise, we follow the lower convex hull of all points in $P$
between $b_i$ and $b_j$, who are above the x-axis and below the line
segment between $b_i$ and $b_j$.

The length of the path between $b_i,b_j$ with no shortcuts is at most
$t + 2(y(b_i) + y(b_j))$ and the length of the path between $b_i,b_j$
using the lower hull is at most $t \cdot \pi /2$ (by Lemma 3).

% diagram showing lower convex hull

$(z_k,z_{k+1})$ is an edge on the lower convex hull between $b_i$ and
$b_j$.  Let $L$ be the lower semi-circle of $circle(z_k, z_{k+1})$.
Without loss of generality, assume $y(z_k) \le y(z_{k+1})$.

\begin{Lemma}

  No points of $P$ are within $L$ and $R_3$.
  
\end{Lemma}

\begin{proof}
  
  From assumption, $L$ and $R_3$ do not intersect.

\end{proof}

\begin{Lemma}

  No points of $P$ are within $L$ and $R_2$.
  
\end{Lemma}

\begin{proof}

  Since $z_k \in P$, it must lie above $T$ since it can't be within $C$,
  therefore $y(z_k) \ge h > w/4$ by the fact that we only build this
  lower hull in the case where $h > w/4$.

  TODO: why is $y(q) \le y(z_k)$?

\end{proof}

\begin{Lemma}

  No points of $P$ are within $L$ and $R_4$.
  
\end{Lemma}

\begin{proof}

  Any point within $L$ and $R_4$ must also be within $C$.

\end{proof}

\begin{Lemma}

  The direct DT path from $z_k$ to $z_{k+1}$ is one-sided.
  
\end{Lemma}

\begin{proof}

  We prove this by showing that $L$ contains no points of $P$.  From
  Lemmas 4, 5, and 6 it remains only to show that there are no points
  of $P$ within $L$ and $R_1$.

  No points of $P$ in $R_1$ exist between the lines $y=h$ and $y=-h$,
  since this area is strictly within $C$.

  No points of $P$ may exist in $R_1$ above $h$ but below the line
  segment $z_kz_{k+1}$, otherwise it would be on the lower convex hull
  and $z_k$ and $z_{k+1}$ would not be adjacent.

\end{proof}

\begin{Lemma}

  Where $a,b,c$ are sides of a right triangle with $c$ being the
  hypotenuse, then

  \begin{displaymath}
    %
    \frac{a}{2} + b \le \frac{\sqrt{5}}{2} \cdot c
    %
  \end{displaymath}
  
\end{Lemma}

\begin{proof}
  Without loss of generality, assume $b \le a$.
  
  \begin{align*}
    %
    a^2 + b^2 &= c^2 \\
    5a^2 + 5b^2 &= 5c^2 \\
    a^2 + 4a^2 + 5b^2 &= 5c^2 \\
    a^2 + 4a^2 + 4b^2 &\le 5c^2 \\
    a^2 + 4ab + 4b^2 &\le 5c^2 \\
    (a + 2b)^2 &\le 5c^2 \\
    a + 2b &\le \sqrt{5} \cdot c \\
    \frac{a}{2} + b &\le \frac{\sqrt{5}}{2} \cdot c \\
    %
  \end{align*}
\end{proof}

\begin{Theorem}

  There exists a DT path from $a$ to $b$ of length

  \begin{displaymath}
    %
    \le \frac{\pi}{2} \cdot (1 + \sqrt{5}) \cdot d(a,b)
    %
  \end{displaymath}

\end{Theorem}

\begin{proof}

  By Lemmas 3 and 7, we know that the length of the shortcut path is
  the sum of the one-sided paths in the lower convex hull between
  $b_i$ and $b_j$ which is at most $t \cdot \pi / 2$.

  In the case where $h \le w/4$ and we don't take a shortcut, then let
  $q$ be the lowest point on $T$, $t$ be the length of $T$, $t_i$ be
  the section of $T$ between $b_i$ and $q$, $t_j$ be defined similarly
  for $b_j$, $w_i$ be the projection of $t_i$ on the x-axis, and $w_j$
  be defined similarly for $t_j$.

  The length of the path is at most

  \begin{align*}
    %
    t + 2(y(b_i) + y(b_j))
    &= t + 2(2h + (y(b_i) - h) + (y(b_j) - h)) \\
    &\le t + 2(\frac{w}{2} + (y(b_i) - h) + (y(b_j) - h)) \\
    &\le t + 2(\frac{w_i}{2} + (y(b_i) - h) + \frac{w_j}{2} + (y(b_j) - h)) \\
    &\le t + 2(\frac{\sqrt{5}}{2} \cdot t_i + \frac{\sqrt{5}}{2} \cdot t_i) \\
    &\le t + 2(\frac{\sqrt{5}}{2} \cdot (t_i + t_i)) \\
    &\le t + \sqrt{5} \cdot t \\
    &\le t (1 + \sqrt{5})
    %
  \end{align*}

  From Lemma 3, we know that half the boundary of the unioned circles
  is an upper bound on the length of the path, so it must be at most
  $\frac{\pi}{2} \cdot (1 + \sqrt{5}) \cdot d(a,b)$.

\end{proof}

\newpage
\part{Keil's Results}

TODO

% Content ends here

\newpage

\bibliography{references} % uncomment line to add references

\end{document}

%%%%%%%%%%%%%%%%%%%%%%%%%%%%%%%%%%%%%%%%%%%%%%%%%%%%%%%%%%%%%%%%%%%%%%
% DelaunayTriangulationSpannerNotes
% March 20th, 2013
% By Simon Pratt (mostly)
\documentclass{tufte-handout}

\usepackage{includes/CGAlgorithms}
\usepackage{includes/QuestionAnswer}
\usepackage{includes/TheoremStuff}
\usepackage{includes/HeaderStuff}
\usepackage{natbib}
\usepackage[pdftex]{graphicx}

%%%%%%%%%%%%%%%%%%%%%%%%%%%%%%%%%%%%%%%%%%%%%%%%%%%%%%%%%%%%%%%%%%%%%%
% Configuration
\newcommand{\DocTitle}{Delaunay Graph Spanner Notes}
\newcommand{\DocAuthor}{Simon Pratt}

\title{\DocTitle}
\author{\DocAuthor}
\fancyhead[L]{\DocTitle}
\fancyhead[R]{\DocAuthor}
\bibliographystyle{plain}

%%%%%%%%%%%%%%%%%%%%%%%%%%%%%%%%%%%%%%%%%%%%%%%%%%%%%%%%%%%%%%%%%%%%%%
% Document
\begin{document}

\maketitle
\vspace{1cm}
% Content starts here

In these notes, we discuss the major results with respect to the
Delaunay Graph as a spanner.

\part{Delaunay Graph}

$P$ is a set of points in the plane, $DG(S)$ is a graph whose vertex
set is $P$ where $u$ and $v$ are connected by an edge only if the
voronoi regions for $u$ and $v$ share an edge.

\begin{figure}
  % http://groups.csail.mit.edu/graphics/classes/6.838/F01/lectures/Delaunay/Delaunay2D.ppt
  \includegraphics{figures/delaunay_graph.png}
  \caption{The Delaunay graph on $P$, including the boundaries of the
    Voronoi regions.}
\end{figure}

\part{Dobkin's Results}

The Delaunay triangulation of a set of points in the plane is a
spanner with spanning ratio $c \le ((1 + \sqrt{5})/2)\pi \approx
5.08$.  This was proven in the paper ``Delaunay Graphs Are Almost as
Good as Complete Graphs'' by Dobkin, Friedman, and Supowit
\cite{Dobkin:1987} \cite{Dobkin:1990}.

\section{Introduction}

We consider the path between two arbitray points $a,b \in P$.  Let the
line connecting $a,b$ be the \emph{direct line}.  We construct
\emph{the direct DT path} by walking along the direct line, each time
a new face of the Voronoi diagram is reache we add the corresponding
edge in the Delaunay Graph.

\section{One-Sided Path: The Easy Case}

If all edges along the direct DT path between points $a,b \in P$ are
either all above or all below the direct line, we say that this is a
one-sided path.

\begin{figure*}
  \includegraphics[scale=1.0]{figures/one-sided_path.pdf}
  \caption{The cyan line shows the direct path, the green line shows
    the direct DT path, the dashed red lines show the boundaries of
    the Voronoi regions, and the circumcircles (also dashed) are
    blue.}
\end{figure*}

% is this necessary for the proof??

\begin{Lemma}

  Points along a direct DT path are monotonic in $x$.

\end{Lemma}

\begin{Lemma}

  All points along the direct DT path from $a$ to $b$ are contained
  within or on the boundary of the circle with $a$ and $b$
  diametrically opposed.
  
\end{Lemma}

\begin{Lemma}

  The boundary of a connected union of circles has boundary at most
  $\pi \cdot (x_r - x_l)$ where $x_r$ and $x_l$ are the extreme x
  coordinates of any of the circles.
  
\end{Lemma}

From these lemmas, it follows that the one-sided path is at most
$\pi/2$ times as long as the euclidean distance between the endpoints.

\section{The Harder Case}

The direct DT path may cross the direct line $\BigOmega{n}$ times,
which can yield a much longer path.

\begin{figure}
  \includegraphics[scale=1.0]{figures/two-sided_path.pdf}
  \caption{A similar diagram with a two-sided direct DT path.}
\end{figure}

\part{Keil's Results}

TODO

% Content ends here

\newpage

\bibliography{references} % uncomment line to add references

\end{document}
